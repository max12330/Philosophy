\documentclass[coursework, och]{SCWorks1}
\usepackage[T2A]{fontenc}
\usepackage[utf8]{inputenc}
\usepackage{graphicx}

\usepackage[sort,compress]{cite}
\usepackage{amsmath}
\usepackage{amssymb}
\usepackage{amsthm}
\usepackage{fancyvrb}
\usepackage{longtable}
\usepackage{array}
\usepackage[english,russian]{babel}

\usepackage[colorlinks=false]{hyperref}

\newcommand{\eqdef}{\stackrel {\rm def}{=}}

\newtheorem{lem}{Лемма}
\newtheorem{definition}{Определение}

\begin{document}
\begin{center}
	Философия
\end{center}

\textbf{Лекция 11.02.2026}

Зачёт: тест по лекциям в конце семестра. автомат --- 55 баллов

По желанию составить глоссарий в электронном виде на почту или распечатать. на доп. баллы.

Кафедра философии и культуры 217к.

Философия с 6 века до н.э.
до 1861 года философия --- любовь к мудрости.

Два основных направления: Наука, мировоззрение.


Наука --- система по выработке объективных знаний о мире и человеке.
Типы философии. Предмет философии. Принципы. Законы. Категории. Методы.

Философия изучает весь мир и место человека в этом мире.

Мировоззрение --- система взглядов на мир в целом и метро человека в этом мире.
Формы мировоззрения: искусство.ю наука, миф, религия, философия.

Основные формы философствования: 

--- Теоретическая: трактаты, статьи, размышления, письма, апологии, диалоги, монографии, диссертации.

--- Художественная: стихи, романы, картины, симфонии, оратории, статуи.

--- Обыденная: рассуждаю о жизни людей, не связанных с изучением преподаванием философией.

Выбор: поспешишь --- людей насмешишь. и под лежачий камень вода не течет.

Картина Джорджоне <<Три философа>>.

\begin{center}
Происхождение философского знания: генезиз философии	
\end{center}

Когда: 5-8 век до н.э.

Где: В странах с развитой цивилизацией (Китай, Индия, Древний Рим).

Из чего: Мифогенная(Гегель) и гносеогенная(Спенсер) концепции.

Почему: Предпосылки: экономические, политические, социокультурные.

\begin{center}
	Специфика философского знания
\end{center}

В. С. Соловьёв: <<Философии до всего есть дело>>

Осмысление общих проблем бытия: жизнь и смерть, добро и зло, пространство и время, мир и человек по взаимосвязи и развитии.

Принципы, понятия, законы, категории.

Язык философии.

\begin{center}
	Система философского знания
\end{center}

Онтология --- учение о бытии, сущем.

Гносеология --- учение о познании.

Эпистемология --- учение о знании.

Логика --- учение о формах мышления.

Этика --- учение о моральных ценностях.

Мораль -- представление о принципаз и нормах желаемого 

Нравственность --- совокупность норм и правил которыми руководствуется человек в своём реальном поведении.


Эстетика --- учение о чувственном восприятии мира. Изучает все чувства.

Аксиология --- учение о ценностях.

Социальная философия --- учение об обществе.

История философии --- история философских учений.

\begin{center}
	Соотношение философии с лругими науками
\end{center}

Связана со всего науками непо

\begin{center}
	Основные функции философии
\end{center}

Функция выделения наиболее общих идей.

Мировоззренческая.

Методологическая.

Гносеологическая.

Критическая.

Прогностическая --- способность делать прогноз на будущее.

Нравственная.

Эвристическая --- связана с именем: Архимед.

Аксиологическая.

\textit{
Почему др. греч. сооружения до сих пор стоят? Можно ли воссоздать технологию?
}


\textbf{Лекция 25.02.2026}

\begin{center}
	Человек в философии Средневековья
\end{center}
<<Дух, душа, тело>> (Ориген). \\
<<Человек слаб и греховен>>, <<Спасенная душа вновь обращается к Богу>> (Августин). \\
<<Верую, дабы понимать>> (Ансельм). \\
<<Гармония веры и разума>> (Фома Аквинский). \\
<<Без необходимости не следует многое утверждать>> (Абеляр). \\

\begin{center}
	Философия Возрождения (ренессанса) (14 -- 17 века)
\end{center}
Гуманизм \\
Пантеизм \\
Проблемы человека (Данте, Петрарка), природы (Бруно, Галилей, да Винчи, Коперник), государства и права (Макиавелли) \\
Критика схоластики \\
Связь философии с наукой и искусством. \\

\begin{center}
	Леонардо да Винчи (1452 -- 1519 гг.)
\end{center}

Живописец, скульптор, архитектор, рисовальщик, ученый экспериментатор, инженер, основоположник культуры Высокого Возрождения. <<Мона Лиза>> --- возвышенный идеал вечной женственности, красоты и обаяния.

Тайная вечеря Леонардо да Винчи (1495 -- 1498 гг.)

\begin{center}
	Вильям Шекспир (1564 -- 1616 гг.)
\end{center}
Сонет 90 (1609 г.)
Самуил Маршак лучший переводчик Шекспира.

\begin{center}
	Становление и развитие философского \\
	рационализма Нового времени (17 -- 18 века)
\end{center}

\begin{itemize}
	\item Рационализм, механицизм, эмпиризм.
	\item Методы познания. роль опыта чувств, разума в познании.
	\item Борьба с теологией.
	\item Ф. Бэкон <<Знание есть сила>>, индукция, эксперимент.
	\item Р. Декарт <<Я мыслю, следовательно, я существую>>, принцип сомнения, дедукция, дуализм.
\end{itemize}
рацио -- разум.

\begin{center}
	Философия французского Просвещения (18 век)
\end{center}

\begin{itemize}
	\item Ф. Вольтер: защитник свободы личности, свободы мысли, ратовал за свободное развитие науки, искусства, философии; двойственное отношение к регилии.
	\item Последователи: Руссо, Дидро, Гольбах, Монтескье, Гельвеции.
	\item Создатели первой в мире <<Экциклопедии>>.
\end{itemize}

\begin{center}
	Теоретическая и практическая философия И. Канта (1724 --- 1804 гг.)
\end{center}
\begin{itemize}
	\item Классический идеалист
	\item Космологическая гипотеза
	\item "Критика чистого разума", Критика практического разума", "Критика способности суждения".
	\item Агностик --- <<вещь в себе>>
	\item "Категорический императив" --- самоценность каждой личности
	\item Основоположник антропологии
	\item Что я могу знать?
	\item Что я должен делать?
	\item На что я смею надеятся?
	\item 
\end{itemize}

%===============================================================================
\end{document}