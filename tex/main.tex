\documentclass[coursework, och]{SCWorks1}
\usepackage[T2A]{fontenc}
\usepackage[utf8]{inputenc}
\usepackage{graphicx}

\usepackage[sort,compress]{cite}
\usepackage{amsmath}
\usepackage{amssymb}
\usepackage{amsthm}
\usepackage{fancyvrb}
\usepackage{longtable}
\usepackage{array}
\usepackage[english,russian]{babel}

\usepackage[colorlinks=false]{hyperref}

\newcommand{\eqdef}{\stackrel {\rm def}{=}}

\newtheorem{lem}{Лемма}
\newtheorem{definition}{Определение}

\begin{document}
\begin{center}
	Философия
\end{center}

\textbf{Лекция 11.02.2026}

Зачёт: тест по лекциям в конце семестра. автомат --- 55 баллов

По желанию составить глоссарий в электронном виде на почту или распечатать. на доп. баллы.

Кафедра философии и культуры 217к.

Философия с 6 века до н.э.
до 1861 года философия --- любовь к мудрости.

Два основных направления: Наука, мировоззрение.


Наука --- система по выработке объективных знаний о мире и человеке.
Типы философии. Предмет философии. Принципы. Законы. Категории. Методы.

Философия изучает весь мир и место человека в этом мире.

Мировоззрение --- система взглядов на мир в целом и метро человека в этом мире.
Формы мировоззрения: искусство.ю наука, миф, религия, философия.

Основные формы философствования: 

--- Теоретическая: трактаты, статьи, размышления, письма, апологии, диалоги, монографии, диссертации.

--- Художественная: стихи, романы, картины, симфонии, оратории, статуи.

--- Обыденная: рассуждаю о жизни людей, не связанных с изучением преподаванием философией.

Выбор: поспешишь --- людей насмешишь. и под лежачий камень вода не течет.

Картина Джорджоне <<Три философа>>.

\begin{center}
Происхождение философского знания: генезиз философии	
\end{center}

Когда: 5-8 век до н.э.

Где: В странах с развитой цивилизацией (Китай, Индия, Древний Рим).

Из чего: Мифогенная(Гегель) и гносеогенная(Спенсер) концепции.

Почему: Предпосылки: экономические, политические, социокультурные.

\begin{center}
	Специфика философского знания
\end{center}

В. С. Соловьёв: <<Философии до всего есть дело>>

Осмысление общих проблем бытия: жизнь и смерть, добро и зло, пространство и время, мир и человек по взаимосвязи и развитии.

Принципы, понятия, законы, категории.

Язык философии.

\begin{center}
	Система философского знания
\end{center}

Онтология --- учение о бытии, сущем.

Гносеология --- учение о познании.

Эпистемология --- учение о знании.

Логика --- учение о формах мышления.

Этика --- учение о моральных ценностях.

Эстетика --- учение о чувственном восприятии мира. Изучает все чувства.

Аксиология --- учение о ценностях.

Социальная философия --- учение об обществе.

История философии --- история философских учений.

\begin{center}
	Соотношение философии с лругими науками
\end{center}

Связана со всего науками непо

\begin{center}
	Основные функции философии
\end{center}

Функция выделения наиболее общих идей.

Мировоззренческая.

Методологическая.

Гносеологическая.

Критическая.

Прогностическая --- способность делать прогноз на будущее.

Нравственная.

Эвристическая --- связана с именем: Архимед.

Аксиологическая.

\textit{
Почему др. греч. сооружения до сих пор стоят? Можно ли воссоздать технологию?
}






%===============================================================================
\end{document}